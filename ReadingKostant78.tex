%%vim latex=xelatex
%% abbreviate jjet U_\eta(\mathfrak n)
%%%%%%%%%%%%%%%%%%%%%%%%%%%%% Define Article %%%%%%%%%%%%%%%%%%%%%%%%%%%%%%%%%%
\documentclass{article}
%%%%%%%%%%%%%%%%%%%%%%%%%%%%%%%%%%%%%%%%%%%%%%%%%%%%%%%%%%%%%%%%%%%%%%%%%%%%%%%

%%%%%%%%%%%%%%%%%%%%%%%%%%%%% Using Packages %%%%%%%%%%%%%%%%%%%%%%%%%%%%%%%%%%
\usepackage[marginparwidth=1.3in, hoffset=0.5in, voffset=-0.5in, textheight=680pt]{geometry}
\usepackage{amssymb}
\usepackage{amsmath}
\usepackage{amsthm}
\usepackage{bm}
\usepackage{empheq}
\usepackage{mdframed}
\usepackage{booktabs}
\usepackage{color}
\usepackage{hyperref}
\usepackage{todonotes}
  \setuptodonotes{noline, color=orange!30}
  \reversemarginpar
\usepackage{fontspec}
  \setmainfont{Libertinus Sans}
% \usepackage[mathrm=sym, usefilenames]{firamath-otf}
\usepackage[mathrm=sym]{unicode-math}
\setmathfont{Libertinus Math}

%%%%%%%%%%%%%%%%%%%%%%%%%%%%%%%%%%%%%%%%%%%%%%%%%%%%%%%%%%%%%%%%%%%%%%%%%%%%%%%

% Other Settings

%%%%%%%%%%%%%%%%%%%%%%%%%% Page Setting %%%%%%%%%%%%%%%%%%%%%%%%%%%%%%%%%%%%%%%
\geometry{a4paper}

%%%%%%%%%%%%%%%%%%%%%%%%%% Define some useful colors %%%%%%%%%%%%%%%%%%%%%%%%%%
\definecolor{ocre}{RGB}{243,102,25}
\definecolor{mygray}{RGB}{243,243,244}
\definecolor{deepGreen}{RGB}{126,211,0}
\definecolor{shallowGreen}{RGB}{235,255,255}
\definecolor{deepBlue}{RGB}{61,124,222}
\definecolor{shallowBlue}{RGB}{235,249,255}
\definecolor{peach}{RGB}{255,164,164}
\definecolor{shallowGray}{RGB}{235,255,235}
\definecolor{tangerine}{RGB}{255,158,26}
\definecolor{shallowYellow}{RGB}{234,206,100}
%%%%%%%%%%%%%%%%%%%%%%%%%%%%%%%%%%%%%%%%%%%%%%%%%%%%%%%%%%%%%%%%%%%%%%%%%%%%%%%

%%%%%%%%%%%%%%%%%%%%%%%%%% Define an peachbox command %%%%%%%%%%%%%%%%%%%%%%%%
\newcommand\peachbox[1]{\fcolorbox{ocre}{peach}{\hspace{1em}#1\hspace{1em}}}
\newcommand\bluebox[2]{\fcolorbox{white}{shallowBlue}{\hspace{1em}\LARGE \textbf{#1}}}
\newcommand\x[1]{\text{#1}}
%%%%%%%%%%%%%%%%%%%%%%%%%%%%%%%%%%%%%%%%%%%%%%%%%%%%%%%%%%%%%%%%%%%%%%%%%%%%%%%

\DeclareMathOperator{\End}{End}
\DeclareMathOperator{\Ker}{Ker}

%%%%%%%%%%%%%%%%%%%%%%%%%%%% English Environments %%%%%%%%%%%%%%%%%%%%%%%%%%%%%
\newtheoremstyle{mytheoremstyle}{3pt}{3pt}{\normalfont}{0cm}{\rmfamily\bfseries}{}{1em}{{\color{black}\thmname{#1}~\thmnumber{#2}}\thmnote{\,--\,#3}}
\newtheoremstyle{myproblemstyle}{3pt}{3pt}{\normalfont}{0cm}{\rmfamily\bfseries}{}{1em}{{\color{black}\thmname{#1}~\thmnumber{#2}}\thmnote{\,--\,#3}}
\theoremstyle{mytheoremstyle}
\newmdtheoremenv[linewidth=1pt,backgroundcolor=yellow,leftmargin=0pt,innerleftmargin=20pt,innerrightmargin=20pt,]{theorem}{Theorem}[section]
\theoremstyle{mytheoremstyle}
\newmdtheoremenv[linewidth=1pt,backgroundcolor=tangerine,leftmargin=0pt,innerleftmargin=20pt,innerrightmargin=20pt,]{definition}{Definition}[section]
\theoremstyle{myproblemstyle}
\newmdtheoremenv[linecolor=black,leftmargin=0pt,innerleftmargin=10pt,innerrightmargin=10pt,]{problem}{Problem}[section]
\newmdtheoremenv[backgroundcolor=yellow!50, linewidth=1pt,leftmargin=0pt,innerleftmargin=10pt,innerrightmargin=10pt,]{lemma}{Lemma}[section]
\newmdtheoremenv[linewidth=1pt,linecolor=black,backgroundcolor=peach,leftmargin=0pt,innerleftmargin=10pt,innerrightmargin=10pt,]{remark}{Remark}[section]
\newmdtheoremenv[linecolor=black,leftmargin=0pt,innerleftmargin=10pt,innerrightmargin=10pt,]{exercise}{Exercise}[section]
\newmdtheoremenv[linecolor=black,leftmargin=0pt,innerleftmargin=10pt,innerrightmargin=10pt,]{example}{Example}[section]
\newmdtheoremenv[linewidth=1pt,backgroundcolor=shallowGreen,leftmargin=0pt,innerleftmargin=10pt,innerrightmargin=10pt,]{fact}{Fact}[section]
%%%%%%%%%%%%%%%%%%%%%%%%%%%%%%%%%%%%%%%%%%%%%%%%%%%%%%%%%%%%%%%%%%%%%%%%%%%%%%%

%%%%%%%%%%%%%%%%%%%%%%%%%%%%%%% Title & Author %%%%%%%%%%%%%%%%%%%%%%%%%%%%%%%%
\title{Kostant 78, On Whittaker Vectors and Representation Theory}
\author{Karthik Dulam}
%%%%%%%%%%%%%%%%%%%%%%%%%%%%%%%%%%%%%%%%%%%%%%%%%%%%%%%%%%%%%%%%%%%%%%%%%%%%%%%

\begin{document}
    \maketitle
    \listoftodos


\section{Section}
    
    Let $\mathfrak g$ be a complex semisimple Lie algebra with Cartan subalgebra $\mathfrak h$ and root system $\Phi$.

    In section 1.1, we defined $\mathfrak n \subset \mathfrak g$ to be the nilradical of 
    $\mathfrak b = \mathfrak h + \sum_{\phi \in \Delta_+} \mathfrak g_\phi$. $\mathfrak n$ is a maximal nilpotent subalgebra 
    of $\mathfrak g$. We also defined $\mathfrak n^-$ to be the nilradical of the opposite Borel subalgebra 
    $\mathfrak b^- = \mathfrak h + \sum_{\phi \in \Delta_-} \mathfrak g_\phi$. \\

    In section 2.3, we defined a Lie algebra homomorphism $\eta: \mathfrak n \to \mathbb C$ to be non-singular if 
    the constants $c_i = \eta(e_{\alpha_i})$ are non-zero for all $\alpha_i \in \Delta_+$. This extends to a homomorphism 
    $\eta: U(\mathfrak n) \to \mathbb C$, and let $U_\eta(\mathfrak n)$ be the kernel of this homomorphism. 
    We also know that $U(\mathfrak n) = \mathbb C \oplus U_\eta(\mathfrak n)$ and since $\mathfrak g = \overline{\mathfrak b} \oplus \mathfrak n$,
    we have $U = U(\overline{\mathfrak b}) \oplus U U_\eta(\mathfrak n)$. And now let $\rho_\eta$ be the projection map
    form $U$ to $U(\overline{\mathfrak b})$ with kernel $UU_\eta(\mathfrak n)$.

    \todo[noline]{Read section 2.5 in paper when needed.}

    \begin{definition}[Whittaker Vector and Whittaker Module]
	A vector $w \in V$ is called a Whittaker vector with respect to $\eta$ if $xw = \eta(x)w$ for all $x \in \mathfrak n$.
	A $U-$module $V$ is called a Whittaker module if it contains a cyclic ($Uw = V$) Whittaker vector.
    \end{definition}

    \begin{remark}
	This definition seems to mimic the ideas from Cartan subalgebra? We have a maximal nilpotent subalgebra and are looking 
	at vectors which are eigenvectors w.r.t. this subalgebra with $\eta$ representing the roots.
	A cyclic Whittaker vector resembles a highest weight vector.
    \end{remark}
    
    If $V$ is a Whittaker module with $w$ as a cyclic Whittaker vector, and if $U_w$ and $U_V$ are the annihilators of $w$ and $V$ 
    respectively, then one has $V \cong U/U_w$ as $U-$modules, $UU_\eta(\mathfrak n) \subseteq U_w$, and $UZ_V \subseteq U_w$ where 
    $Z_V = Z \cap U_V$. This means that $U_w$ contains the kernel of the projection map $\rho_\eta$ and hence $U_w$ is stable.

    Also note $U_w$ is a left ideal and $U_V$ is a two-sided ideal.

    \begin{remark}
    Since $V$ is cyclic, we have $V \cong U/U_w$ as $U-$modules, so $V$ is determined upto equivalence by $U_w$.
    \label{remark:V-determined-by-U-w}
    \end{remark}
    

    \begin{theorem}[$U_w$ decomposition]
    \label{thm:U_w_decomposition}
	 Let V be any $U-$module which admits a cyclic Whittaker vector $w$ then, $U_w = UZ_V + UU_\eta(\mathfrak n)$. 
    \end{theorem}

    To prove this we will need the following lemma. 

    \begin{definition}[$\eta$-reduced action]
    If $x \circ v^\eta = (xv)^\eta$ then the $\eta$-reduced action of $\mathfrak n$ on $U(\overline{\mathfrak b})$ is given by
    $x \cdot v = x \circ v - \eta(x)v$.  
    \end{definition}
    \todo[noline]{Read 2.6 in paper}
	But before that, as mentioned earlier, $U_w$ is stable under the projection map $\rho_\eta$. And for $X \subseteq U$, 
	denote $X^\eta = \rho_\eta(X)$. From section 2.4.11 in the paper, we know that $\rho_\eta$ induces an isomorphism 
	$Z \to W(\overline{\mathfrak b}) = Z^\eta$.  
	\todo[noline]{Read 2.4.11 in paper} 
	If $Z_*$ is an ideal in $Z$, then $W_*(\overline{\mathfrak b}) = Z_*^\eta$ is an isomorphic ideal in $W(\overline{\mathfrak b})$.
	Thus, we have $(UZ_*)^\eta = U(\overline{\mathfrak b}) W_*(\overline{\mathfrak b}) = \tilde A \otimes W_*(\overline{\mathfrak b})$.
	This is due to Remark 2.3 in paper which says $(uv)^\eta = u^\eta v^\eta$, \todo[noline]{Read Remark 2.3 in paper}
	and Theorem 2.5 in paper which gives the last equality.  \todo[noline]{Read Theorem 2.5 in paper, what the hell is happening over there?}
	And finally by 2.3.5 we have
  \begin{equation}
  UZ_* + UU_\eta(\mathfrak n)  = (\tilde A \otimes W_*(\overline{\mathfrak b})) \oplus UU_\eta(\mathfrak n)
    \label{eq:UZ_*}
  \end{equation}


    \begin{lemma}[Technical Lemma]
      Let $X = \{v \in U(\overline{\mathfrak b}) \mid (x \cdot v)w = 0 \text{ for all } x \in \mathfrak n\}$. Then 
      $X = (\tilde A \otimes W_V(\overline{\mathfrak b})) \oplus UU_\eta(\mathfrak n)$ where $W_V(\overline{\mathfrak b}) = (Z_V)^\eta$.
      We also have $U_w(\overline{\mathfrak b}) \subseteq X$ and $U_w(\overline{\mathfrak b}) = \tilde A \otimes W_V(\overline{\mathfrak b})$ 
      where $U_w(\overline{\mathfrak b}) = U_w \cap U(\overline{\mathfrak b})$.

    \end{lemma}

    \begin{proof}[Proof of lemma]
      The proof is technical, best to come back to them later and focus now on what the statement is saying.
    \end{proof}

    \begin{proof}[Proof of theorem]
    	Theorem \ref{thm:U_w_decomposition} says that $U_w$ can be decomposed into two parts, a trivial 
	part $UU_\eta(\mathfrak n)$ and the other $UZ_V$. The content lies in proving 
	$U_w \subseteq UZ_V + UU_\eta(\mathfrak n)$, but that by \eqref{eq:UZ_*} is equivalent to proving 
	"If $u\in U_w$ then $v \in \tilde A \otimes W_V(\overline{\mathfrak b})$ where $v = u^\eta$", the rest 
	follows from the lemma.

    \end{proof}


  \begin{remark}
    \hspace{1cm}
    \begin{itemize}
      \item Theorem \ref{thm:U_w_decomposition} implies, along with remark \ref{remark:V-determined-by-U-w},
	  that a Whittaker module $V$ is determined upto equivalence by the central ideal $Z_V$.
      \item Note that the proof depends on non-singularity of $\eta$. It relies on Theorem 2.5 in the paper, 
	which in turn relies on 2.4.3 in Theorem 2.4.1 in the paper, the part which assumes non-singularity of $\eta$. 
    \end{itemize}
  \end{remark} \todo[noline]{Read 2.4.1 in paper}
    

    \begin{theorem}[Relation between Whittaker modules and ideals of the Center of $U$]
      Let $U, V, U_V, Z_V$ be as above. Then the correspondence 
      \begin{equation}
      	V \mapsto Z_V
      	\label{eq:correspondence}
      \end{equation}
      is a bijection between the set of equivalence classes of Whittaker modules and the set of ideals of $Z$.
    \end{theorem}
    
    \begin{proof}[Proof]
    	Injectivity is clear be Theorem \ref{thm:U_w_decomposition}. 
	For surjectivity, let $Z_*$ be an ideal of $Z$.
	Then we would like to construct a Whittaker module $V$ such that $Z_V = Z_*$. 
	But we already know that $Z_V$ determines $V$ upto equivalence. 
	So let $L = UZ_* + UU_\eta(\mathfrak n)$, then $V = U/L$ is a Whittaker module with $U_w = L$.
    \end{proof}

    \begin{remark}
    	This also gives us a correspondence between the set of ideals of center and the set of annihilators of Whittaker modules.
	$Z_* \mapsto U_w = UZ_* + UU_\eta(\mathfrak n)$.
    \end{remark}
    
    \begin{fact}
      The subalgebra $ZU(\mathfrak n)$ can be written as $Z \otimes U(\mathfrak n)$. 
      \footnote{Reference: Section 3.3 in paper, Theorem 0.12 in \emph{Lie Group Representations on Polynomial Rings by Kostant}}

      Then we can write $Z/Z_*$ as a $Z$-module and thus,
      $(Z/Z_*)_\eta$ as a $Z\otimes U(\mathfrak n)$-module, if $u \in Z, v \in U(\mathfrak n)$ 
      and $y \in (Z/Z_*)_\eta$, then $uvy = \eta(v)uy$.
    \end{fact}

    \begin{theorem}[Every Whittaker module is an induced module]
      Let $V$ be any $U$-module then, $V$ is a Whittaker module if and only if we have the isomorphism 
      $V \cong U \otimes_{Z \otimes U(\mathfrak n)} Z/Z_*$ of $U$-modules for some ideal $Z_*$ of $Z$.
      And furthermore, in such a case $Z_*$ is unique, and $Z_* = Z_V$.
    \end{theorem}
    
    \begin{proof}[Proof]
      \todo{Write proof, not hard, not interesting}
    \end{proof}

    \begin{remark}
    	This resembles the construction using Verma modules.
    \end{remark}
    

    \todo{Define Universal Whittaker Module}

    \begin{theorem}[Whittaker Vectors in a Whittaker Module]
      Let $V$ be any $U$-module with a cyclic Whittaker vector $w$. Then any $v \in V$ is a Whittaker vector if and only if 
      $v$ is form $v = uw$ where $u \in Z$.
    \end{theorem}

    \begin{remark}
    	This theorem neatly characterizes Whittaker vectors in a Whittaker module. 
	And in other words it says that the space of Whittaker vectors is a $Z$-cyclic module 
	isomorphic to $Z/Z_Vw$ since we have a $Z$-module homomorphism $u \in Z \mapsto uw \in V$ with kernel $Z_V = Z_w$ 
	and image being the space of Whittaker vectors.
    \end{remark}
    
    For any $U$-module V, let $\End_U V$ be the algebra of operators on $V$ commuting with $U$,
    and if $\pi_V: U \to \End V$ is a representation of $U$ on $V$, then 
    $\pi_V(Z) \subseteq \End_U V$ and $\pi_V(Z) \cong Z/Z_V$ as algebras.
  
    \begin{theorem}[$\End_U V$ in a Whittaker Module is a commutative algebra]
      Let $V$ be a Whittaker module, then $\End_U V = \pi_V(Z)$ and $\End_U V \cong Z/Z_V$ as algebras.
      And thus $\End_U V$ is a commutative algebra.
    \end{theorem}
    \begin{proof}[Proof]
      Let $w$ be a cyclic Whittaker vector, and if $\alpha \in \End_U V$, then $\alpha w$ is also a Whittaker vector.
      But by the previous theorem, $\alpha w = uw$ for some $u \in Z$. Now for any $v \in U$, we have 
      $\alpha vw = v\alpha w = vuw = uvw$ thus $\alpha = \pi_V(u)$.
    \end{proof}

    \todo{Read remark 3.5 in paper}
    \begin{remark}
      This resembles the result from Dixmier's book on enveloping algebras, that the centralizer of a regular 
      element is commutative. Probably why Prof. Michael was excited about that.
    \end{remark}
    
    We will refer to a homomorphism $\zeta: Z \to \mathbb{C}$ as a central character.
    Let $Z_\zeta = \Ker \zeta$, since image of $\zeta$ is a field, $Z_\zeta$ is a maximal ideal of $Z$.
    Using 2.3.2 in paper, we can naturally parametrize the set of central characters by $\mathbb{C}^l$.
    \todo{Read 2.3.2 in paper}

    If $V$ is a $U$-module, we say $V$ admits an infinitesimal character $\zeta$ if there exists a central character $\zeta$
    if $\zeta$ is a central character such that $uv = \zeta(v)$ for all $u \in Z, v \in V$.
    By Dixmier's Theorem any irreducible $U$-module admits an infinitesimal character.
    \todo{Find reference}

    For a central character $\zeta$, let $\mathbb{C}_{\zeta,\eta}$ be the 1-dimensional $Z\otimes U(\mathfrak n)$-module 
    defined so that if $u\in Z, v \in U(\mathfrak n)$ and $y \in \mathbb{C}_{\zeta,\eta}$, then $uvy = \zeta(u)\eta(v)y$.
    Also let $Y_{\zeta,\eta} = U \otimes_{Z \otimes U(\mathfrak n)} \mathbb{C}_{\zeta,\eta}$.\\

    $Y_{\zeta,\eta}$ admits an infinitesimal character and that is $\zeta$.
    \todo{Why is this?}

      Let $\eta_0$ be the trivial character, and let $Y_{\zeta,\eta_0}$ be defined the same way.
      And let $\lambda: \mathfrak{h} \to \mathbb{C}$ be a linear functional, then it defines a 
      homomorphism $\lambda: U(\mathfrak{h}) \to \mathbb{C}$ and the corresponding Verma module is
      $M_\lambda = U \otimes_{U(\mathfrak{h}) \otimes U(\mathfrak{n})} \mathbb{C}_\lambda$
      where $U(\mathfrak{h})$ acts on $\mathbb{C}_\lambda$ via $\lambda$ and $U(\mathfrak{n})$ acts trivially.
      Now one knows that $M_\lambda$ admits an infinitesimal character $\zeta = \zeta(\lambda)$. 
      \todo{Why? Verma modules are not irreducible. Harish-Chandra isomorphism?}
      And hence considering the cyclic generator, one has the exact sequence
      $Y_{\zeta,\eta_0} \to M_\lambda \to 0$ of $U$-modules.
      \todo{Does this not mean that the map is just surjective?}
      But $M_\lambda$ is in general not irreducible, thus in particular $Y_{\zeta,\eta_0}$ is not in general irreducible.
      In contrast the next theorem asserts that $Y_{\zeta,\eta}$ is always irreducible when $\eta$ is non-singular.

    \begin{theorem}[Big Theorem]
      Let $V$ be any Whittaker module for $U$, the universal enveloping algebra of a semisimple Lie algebra $\mathfrak{g}$, 
      then the following are equivalent:
      \begin{enumerate}
      	\item $V$ is an irreducible $U$-module.
	\item $V$ admits an infinitesimal character.
	\item The corresponding ideal $Z_V$ of $Z$, given by Theorem \ref{thm:U_w_decomposition} is a maximal ideal.
	\item The space of Whittaker vectors in $V$ is 1-dimensional.
	\item All non-zero Whittaker vectors in $V$ are cyclic.
	\item The centralizer $\End_U V$ reduces to the constants.
	\item $V$ is isomorphic to $Y_{\zeta,\eta}$ for some central character $\zeta$.
      \end{enumerate}
    \end{theorem}

    \begin{theorem} 
    	Let $V$ be any $U-$module which admits an infinitesimal character. Assume $w \in V$ is a Whittaker vector, then 
	the submodule $Uw$ is irreducible. If $\zeta$ is the infinitesimal character and $w_\zeta$ is the cyclic 
	Whittaker generator of $Y_{\zeta,\eta}$, then $Uw \cong Y_{\zeta,\eta}$ as $U$-modules with $uw_\zeta \mapsto uw$.
	This isomorphism is unique up to a scalar multiple.

	If $V_1$ and $V_2$ are irreducible $U$-modules with the same infinitesimal characters, and if they have 
	Whittaker vectors, they are unique up to a scalar multiple and $V_1 \cong V_2$ as $U$-modules.
    \end{theorem}

    If $V$ is a Whittaker module, $Z_V$ its corresponding ideal, and if $V_* \subseteq V$ a submodule, then 
    let $Z(V_*) = \{u \in Z \mid u(V) \subseteq V_*\}$ so that $Z(V_*)$ is an ideal in $Z$ containing $Z_V$.

    \begin{remark}
    	If $w \in V$ is a cyclic Whittaker vector, and if $u \in Z$, then $u \in Z(V_*)$ if and only if $uw \in V_*$.
    \end{remark}
    
    \begin{theorem}
    	Let $V$ be a Whittker module, then the map $V_* \to Z(V_*)$ is a bijection between the set 
	of submodules of $V$ and the set of ideals of $Z$ containing $Z_V$. Furthermore, 
	$Z(V_1 + V_2) = Z(V_1) + Z(V_2)$ and $Z(V_1 \cap V_2) = Z(V_1) \cap Z(V_2)$ and the 
	inverse to the correspondece is given by $Z_* \to Z_*V$ for any ideal $Z_*$ of $Z$ containing $Z_V$.
    \end{theorem}

    \begin{remark}
    	While we had the correspondece between $Z_V$ an ideal in $Z$ and $V$ a Whittker module,
	this theorem tells us that ideals $Z_*$ of $Z$ containing $Z_V$ correspond to 
	$U-$submodules $V_* = Z_*V$ of $V$ and the ideal corresponding to $V_*$ is $Z(V_*)$.
	This correspondece is an order reversing lattice isomorphism similar to what we see in 
	Galois theory.
    \end{remark}
    
    \begin{theorem}[Corollary]
      Let $Y_\eta$ be the induced $U-$module denoted by $Y_\eta = U \otimes_{U(\mathfrak{n})} \mathbb{C}_\eta$.
      The action of $Z$ on $Y_\eta$ induces an isomorphism $Z \to \End_U Y_\eta$.
      Furthermore if $Z_*$ is any ideal in $Z$ the map $Z_* \to Z_*Y_\eta$ sets up a bijection of the sets 
      of all ideals in $Z$ and the set of all $U-$submodules of $Y_\eta$.
    \end{theorem}

    \begin{remark}
    	This is just the earlier theorem applied to the universal Whittaker module.
    \end{remark}
    
    Let $\lambda \in \mathfrak{h}'$ be a linear functional on $\mathfrak{h}$ and let $M_\lambda$ be the corresponding
    Verma module. Let $C \subseteq \mathfrak{h}'$ be the $\mathbb{Z}$-cone spanned by the negative roots and zero and 
    let $C_\lambda = \lambda + C$. If $M_\lambda(\mu) \subseteq M_\lambda$ is the weight space corresponding to the weight
    $\mu \in \mathfrak{h}'$, then one has the direct sum $M_\lambda = \bigoplus_{\mu \in C_\lambda} M_\lambda(\mu)$.
    And if $\phi \in \Delta$ one knows that $e_\phi M_\lambda(\mu) \subseteq M_\lambda(\mu + \phi)$.

    Let $U(\mathfrak{n})$ be a $\mathfrak{b}$ module defines as $x \cdot u = x u$ for $x \in \mathfrak{n}$ and 
    $u \in U(\mathfrak{n})$ and $x \cdot u = [x u] $ for $x \in \mathfrak{h}$ and $u \in U(\mathfrak{n})$.
    Let $U(\mathfrak{n})'$ be the dual module. 

    \begin{definition}[Restricted dual to $U(\mathfrak{n})$]
      Let $U(\mathfrak{n})^*$ be the subspace of all $\beta \in U(\mathfrak{n})'$ such that $\beta$ vanishes on a power of
      the augmentation ideal $\mathfrak{n}U(\mathfrak{n})$. Then $U(\mathfrak{n})^*$ is a $U(\mathfrak{b})$-module.
      We refer to $U(\mathfrak{n})^*$ as the restricted dual to $U(\mathfrak{n})$.
    \end{definition}

    Let $0 \neq m_\lambda \in M_\lambda$ be a highest weight vector and $l$ be a linear functional 
    on $M_\lambda$ such that $l(m_\lambda) = 1$ and $l$ vanishes on $M_\lambda(\mu)$ for $\mu \neq \lambda$.
    For any $v \in M_\lambda$ let $\beta_v \in U(\mathfrak{n})'$ be defined by 
    $\langle \beta_v, u \rangle = l(\check uv)$ for $u \in U(\mathfrak{n})$.
    Where $\check u$ denotes the image of the linear isomorphism defined by $\check{wu} = \check u \check w$ 
    and $\check x = -x$ for $x \in \mathfrak{g}$.

    \begin{lemma}
    	The correspondece $v \mapsto \beta_v$ defines a map $M_\lambda \to U(\mathfrak{n})^*$ which is a 
	$\mathfrak{n}$-module homomorphism. And this is an isomorphism of $\mathfrak{n}-$modules iff 
	the Verma module $M_\lambda$ is irreducible.
    \end{lemma}

\end{document}
